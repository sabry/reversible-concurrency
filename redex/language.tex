\documentclass[11pt]{article}

\usepackage{mathpartir}
\usepackage{amsmath}

\newcommand{\bnfdef}{::=}
\newcommand{\bnfalt}{\mid}
\newcommand*{\defeq}{\stackrel{\text{def}}{=}}

\begin{document}

\section{Introduction}
The language we're trying to develop is a simple parallel language, with
synchronous communication, and the ability to backtrack. A process should
be able to perform a speculative computation, and backtrack when some
process decides the computation should be restarted with new values, or
a new computation should be done altogether. All communication that
happened between the start of the speculative computation and the
backtrack signal should be undone, perhaps causing other processes to
backtrack as well.

We've approached this problem by creating choice points, each with
multiple choices. When a process decides to backtrack to a choice point,
it restores the continuation present at the time the choice point was
created, using the next choice in the sequence of choices provided to
fill the continuation.

A choice point might look like:
$$E[\text{choose}~k~e_1~e_2]$$

In this computation, first $E[e_1]$ would be performed. If this process
backtracks to this choice point, it would then perform $E[e_2]$.

A process could backtrack via $$E[\text{backtrack}~k]$$.

\section{Current Language}
The current language as modeled in redex is a simple scheme-like
language, extended to a parallel language with synchronous communication,
choice points, `failing backtracking', sync points, and commit points.

\subsection{Syntax}
\noindent
$
  I \bnfdef nat\\
  P \bnfdef S;I;e \bnfalt P \parallel P \\
  S \bnfdef \cdot \bnfalt S,k \to e \\
  e \bnfdef v \bnfalt delta_1~e \bnfalt delta_2~e~e \bnfalt e~e \bnfalt
  error~string \bnfalt let~x~=~e~in~e \bnfalt choose~k~e~\dots \bnfalt
  seq~e~\dots
  \bnfalt send~I~v \bnfalt recv~I \bnfalt backtrack~I~k~e \bnfalt sync~I
 ~k~e \bnfalt commit~k \\
  v \bnfdef \lambda x.e \bnfalt nat \bnfalt string \bnfalt bool
  \bnfalt () \\
  delta_1 \bnfdef add1 \bnfalt sub1 \bnfalt iszero \\
  delta_2 \bnfdef + \bnfalt - \bnfalt < \\ 
$
$I$ is a process ID, S is a local continuation store.

\subsection{Parallelism}
The language is extended to be parallel by creating a reduction for a
simple scheme-like language, and defining a new reduction as follows:

$$
\inferrule
  {e_1 \rightarrow e_1^\prime \+
   e_2 \rightarrow e_2^\prime}
  {e_1 \parallel e_2 \Rightarrow e_1^\prime \parallel e_2^\prime}
$$

That is, the parallel reduction is simply defined by allowing each
process to perform the small-step reduction in parallel. 

\subsection{Synchronous communication}
The parallel language is extend with two primitive forms that allow
synchronous communication. A sample reduction rule follows:

$$
I_1;E_1[\text{send}~I_2~v] \parallel I_2;E_2[\text{recv}~I_1]
\Rightarrow
I_1;E_1[()] \parallel I_2;E_2[v]
$$

\subsection{Choice points}
A choice points creates a continuation with a hole that can be filled
with one of a sequence of expressions. The continuation can be invoked with
the next expression in the sequence by backtracking to it. Any process
can at anytime try to force another process to backtrack, including
itself. This attempt will succeed if and only if the process being
forced to backtrack has not `commited' (see \ref{subsec:commit}). 

A choice point will cycle through the sequence of choices as long as the
process backtracks to the choice points, and as long as the choice
selected choice is not an error. If the selected choice causes an error,
or is an error statement, of course the cycle will stop with the process
signaling an error.

Choice points create a new mapping from `k-vars' (a unique name for
the continuations) to continuations and place them in a process-local
store.

$$
  S;I;E[\text{choose}~k~e_1~e_2~\dots\parallel P_2 \dots \Rightarrow
  S,k\to;I;E[\text{choose}k~e_2~\dots e_1];I;E[e_1] \parallel P_2 \dots$$

\subsection{Failing backtracking}
Any process can causes another to backtrack at anytime. In previous
version of the language/model, this always causes the target process to
backtrack. In this version, however, if the target process has
committed, the backtrack will fail. The backtrack form thus now requires 
an alternate expression, to be executed if the backtrack fails.

To target a process and force it to backtrack, it's necessary to specify
the process ID and the k-var denoting where the process should backtrack
to.

$$
S_1;I_1;E[\text{backtrack}~I_2~k_1~e_1] \parallel S_2,k_1 \to
E_2[e];I_2;e_2
\parallel P_3 \dots
\Rightarrow S_1;I_1;E[()] \parallel S_2,k_1 \to E_2[e];I_2;e \parallel
P_3 \dots$$

$$S_1;I_1;E[\text{backtrack}~I_2~k_1~e_1] \parallel S_2;I_2;e_2
\parallel P_3 \dots
\Rightarrow S_1;I_1;E[e_1] \parallel S_2;I_2;e_2 \parallel P_3 \dots$$

$\text{if}~k_1~\not\in~S_2$

\subsection{Sync points}
Intuitively, sync points allow two process to synchronize, and specify
some expression to be executed if the process backtracks to a specified
choice point. We used sync points in an earlier model which lacked
choice points, but allowed backtracking to a point without a new choose
to execute. It's not obvious if we still need sync points, but they seem
to provide a nice abstraction, so I've left them in. They also seem to
give a bit of expressiveness that we can't from the other primitives.

Sync points are implemented by modified the stored continuation and
tacking on the expression to execute at the front of the continuation.
Thus, if the sync point is executed, and the process backtracks, it will
perform some extra work before executing the continuation at it's choice
point.

$$
S_1,k_1 \to e_1;I_1;E_1[\text{sync}~I_2~k_1~e_3] \parallel 
S_2,k_2 \to e_2;I_2;E_2[\text{sync}~I_1~k_2~e_4] \parallel P_3 \dots
\Rightarrow $$$
S_1,k_1 \to (\text{begin}~e_3~e_1);I_1;E_1[()] \parallel 
S_2,k_2 \to (\text{begin}~e_4~e_2);I_2;E_2[()] \parallel P_3 \dots
$

\subsection{Commit points}
\label{subsec:commit}
A commit point allows a process to remove all continuations from it's
store prior to a specified point. By treating k-vars as having a linear
order, then given $k$, all mappings $k_1 \to e_1, \dots k_n \to e_n$, where
$k1 < k_n < k$, are removed from the local store. This effectively means
the process will never backtrack to a point prior to $k$. If another
process tries to force the committed process to do so, the backtrack
will fail and the targeting process will take it's alternate path.

$$
S,k_1 \to e_1,S^\prime;I;E[\text{commit}~k_1] \parallel P_1 \dots
\Rightarrow S^\prime;I;E[()]
$$

%\section{Other work}
%Prior to trying to formalize the model, we hacked out an implementation
%in Haskell using true parallelism. This model was pretty broken, and
%what did work was prone to race-conditions. The code, and spare
%documentation, is still available in the repository.
%
%\section{Past Models}
%There are also numerous previous versions of the redex model, each
%toying with one of the above features in a different way. Some versions
%tried to use delimited continuations to model choose and backtrack. Some
%version use a checkpoint, with no choices associated with it, instead of
%a choice point. Others are simply lacking one of the above features,
%such a communication, or sync points, or both. 
%
%Each of these version exist in a branch to themselves. They were not
%actually developed and properly maintaining in seperate branch, but
%ripped out of the master branch and put in their own branches for
%reference purposes. 
%
%They also lack as much documentation as the current version, it may not
%be worth reading them until you've read the current code.


\end{document}
